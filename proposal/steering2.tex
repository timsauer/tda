\documentclass[11pt]{article}
%\usepackage{latexsym,times,epsf}
\usepackage{latexsym,amsmath,amsrefs,amsfonts,amssymb,graphicx}
\usepackage[small,bf]{caption}
\usepackage[large]{subfigure}
%\pagestyle{headings}
%\pagestyle{empty}
\setlength{\topmargin}{-.6in} \setlength{\textheight}{9in}
\setlength{\textwidth}{6.5in} \setlength{\oddsidemargin}{0in}
\def\bc{\begin{center}}
\def\ec{\end{center}}
\def\be{\begin{enumerate}}
\def\ee{\end{enumerate}}
\def\r{\rho}
\def\h{\Theta}
\def\Bbb{\bf}
\def\urltilda{\kern -.15em\lower .7ex\hbox{\~{}}\kern .04em}
\renewcommand{\refname}{}
\usepackage{indentfirst}

\addtolength{\abovecaptionskip}{0mm}
\addtolength{\belowcaptionskip}{0mm}
\addtolength{\floatsep}{-1mm}
\addtolength{\intextsep}{-1mm}
\addtolength{\textfloatsep}{-5mm}

%\input amssym.def
%\input amssym

\usepackage{sectsty}
\sectionfont{\large}
\subsectionfont{\normalsize}

\setlength{\captionmargin}{10pt}
\begin{document}
\renewcommand{\rightmark}{Project Description}
%\thispagestyle{plain}


\bc \Large\bf Section: \\ \Large Pattern identification and steering for \\ spatio-temporal and network dynamics
\ec
%\bc Timothy Sauer\\Department of Mathematics \\ George Mason
%University \\ Fairfax, VA 22030\\ tsauer@gmu.edu
%\ec
\bc \large\bf Project Summary \ec

\begin{itemize}
	\item {\bf Background}: Attractor clustering for pattern detection
	\begin{itemize}
		\item DMDC
		\item CkNN for attractor clustering
		\item Application to Liquid Crystal data
		\item Preliminary study of geometric pattern identification (dim/vol)
	\end{itemize}
	\item {\bf Observability}: Extension of DMDC to neuron data and spike trains
	\begin{itemize}
		\item Introduce observability condition number for Takens embedding theorem
		\item Empirical estimation of observability conditioning?
		\item Improving conditioning through convolution/low-pass filtering?
	\end{itemize}
	\item {\bf Pattern identification}
	\begin{itemize}
		\item Step 1: {\bf Attractor Discrimination}: DMDC+CkNN reconstructs/separates attractors
		\item Step 2: {\bf Pattern Identification}:  Separate attractors may represent similar \emph{patterns}
		\begin{itemize}
			\item {\it Topological}: Attractors with the same topology have a similar dynamical constraints at a very coarse level
			\item {\it Geometric}: Attractors with the same geometry only differ in how the pattern evolves dynamically
			\item {\it Dynamical}: Attractors with the same dynamical properties (Lyapunov spectrum and stochastic forcing) represent identical patterns
		\end{itemize}
		\item Step 3: {\bf Steering}: Move to nearest attractor with desired pattern
		\begin{itemize}
			\item Transition probabilities
			\item Basin identification for pattern classes?
			\item State space exploration?
		\end{itemize}
	\end{itemize}
	\item {\bf Application to Spatiotemporal and Network Dynamics}
	\begin{itemize}
		\item {\bf Diffusion distance}: Hierarchical metric using a \emph{dictionary} built from data subsets which are localized in the spatial or network structure.  Can be designed to be invariant to spatial/network transformations (translation, rotation, ect.).
		\item {\bf Nematic Liquid Crystals}: 
		\item {\bf Neuronal Networks}: 
	\end{itemize}

\end{itemize}


\newpage

\setcounter{page}{1}

\bc \large\bf  Section: \\ Pattern identification and steering for \\ spatio-temporal and network dynamics

\ec

\section{Background and motivation}



\subsection{Preliminary research: Attractor clustering for pattern detection}



\bc \Large\bf References Cited \ec
\setcounter{page}{1}
\bibliographystyle{plain}
\bibliography{bigdata}


%\setlength{\textwidth}{5.9in}
%\setlength{\oddsidemargin}{.3in}



\end{document}